%\selectlanguage{npolish}
\renewcommand{\headline}{Gra w $\cdot$ Pentagram - Polski}
\renewcommand{\tocent}{Polski}
\renewcommand{\translator}{E.A.Szyszka}
\renewcommand{\general}{
Gra, którą pojmiesz w kilka minut, a będzie Cię fascynować latami.
W dwóch graczy jedna partia gry trwa jedynie 20 minut. W trzy lub cztery osoby, gra potrwa około 40-90 minut. 
Gra nie zawiera kości, jest odpowiednia dla graczy od 5 roku życia. Stworzona przez Jana  \noun{Suchanka.}
Pudełko zawiera: 4$\times$5 kolorowe pionki, 5 czarnych i 5 szarych pionków oraz plansza do gry.
}

\renewcommand{\choosext}{Pionki}
\renewcommand{\choosex}{
Każdy z graczy ma pod sobą drużynę składającą się z pięciu pionków.
Każdy wybiera drużynę.
W każdej drużynie jest po jednym pionku koloru niebieskiego, czerwonego, białego, zielonego i żółtego.
Pionki zaczynają grę na pięciu rogach planszy o odpowiadających kolorach. Celem gry jest dotarcie do pól na środku planszy o kolorach odpowiadającym pionkom. 
}

\renewcommand{\setupt}{Przygotowanie}
\renewcommand{\setup}{
Pionki są ustawione na polach po zewnętrznej części planszy w taki sposób, by kolory pól odpowiadały kolorom pionków: biały pionek na białym narożniku, niebieski pionek na niebieskim narożniku itd. 
Czarne pionki są umieszczone na każdym skrzyżowaniu w środku planszy.
Na początku szare pionki pozostają umieszczone na środku planszy.
}

\renewcommand{\objectivet}{Cel gry}
\renewcommand{\objective}{
Białe pionki muszą dostać się do białych skrzyżowań na środku planszy, niebieskie pionki muszą dotrzeć do niebieskich skrzyżowań itd. Celem pionków jest dotarcie do centralnych pól na przeciwko pola, z którego startuje pionek.
Kto pierwszy swoje \emph{trzy} pionki ustawi na odpowiednich środkowych polach, wygrywa.  
}

\renewcommand{\rulest}{Zasady gry}
\renewcommand{\rules}{
Przeciągnij jeden z pionków na skrzyżowanie lub koło w dowolnym kierunku, tak daleko, jak to tylko możliwe.
Możesz skręcić na na dowolnym wolnym skrzyżowaniu bez zatrzymywania się.
Możesz przeciągnąć pionek przez kilka pustych pól, ale nie możesz skakać przez pionki! 
\myskip

Można przejść do pola, który jest zajęty i trafiony:
\myskip

Jeśli trafisz w czarny pionek, umieść go na dowolnym wolnym polu i utwórz blokadę. 
Jeśli trafisz w pole, na którym stoi pionek, zamień pozycjami oba pionki.

\myskip

W ten sposób możesz również zamienić dwa własne pionki.
Jeśli przeniesiesz się na pole, na którym jest kilka pionków, musisz zamienić się z \emph{jedną} z nich..
 
\myskip 
 
Nie możesz wykonać tego samego ruchu dwa razy. 

\myskip 

Kiedy pionek dotrze do celu, jest on wyłączony z gry. Pionki wyłączone z gry ustaw w centrum planszy. 
Jeśli nastąpisz na pole z szarym pionkiem, należy odłożyć szary pionek na bok i wyłączyć go z gry.


\myskip

Ten, kto pierwszy ustawi \emph{trzy } spośród swoich pionków na odpowiednich polach, wygrywa. 

\myskip

Nadmierna gadatliwość podczas gry eliminuje zawodnika.
}