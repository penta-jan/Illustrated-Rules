\renewcommand{\headline}{Penta$\cdot$ludus - Latine}
\renewcommand{\tocent}{Latine}
\renewcommand{\translator}{Dr.~J\"org Grimm / J.S.}
\renewcommand{\general}{ 

%\textbf{Descriptio }
Tabula pentagrammiforma, quattuor cohortes quinorum peditum coloratorum, quina impedimenta nigra et cana Penta\-ludum componunt. Fit Ioh.~\noun{Aridus.}

%\textbf{Cohortes }





}
\renewcommand{\choosext}{Cohortes}
\renewcommand{\choosex}{
Lusor quisque agit pedites quinos, quibus est eadem forma (aut stella aut luna \&c.), sed sunt diversi colores. 

Sic tua cohors peditem caeruleum, peditem rubrum  \&c. continet.

}

\renewcommand{\setupt}{Dispositio} 
\renewcommand{\setup}{ 
In quinque nodis positis in circulo, qui circumvenit pentagramma, ponite pedites eiusdem coloris: pedites albos in nodo albo, caeruleos
in nodo caeruleo \&c. 

Praeter illa decem loca quae nodos formant, existunt alter octoginta aedicula in viis, ubi pedites in ludo etiam possunt collocari. 

Sunt etiam impedimenta nigri et im\-pe\-di\-men\-ta cana. 

Impedimenta nigra ponite in nodis centri. In centro retinete impedimenta cana. 
}

\renewcommand{\objectivet}{Propositum} 
\renewcommand{\objective}{ 

Destinatio per unum quemque peditem, qui occupat nodum externi circuli, est nodus eiusdem coloris in interno pentagrammatis. 
Albo est eundum ad album, rubro ad rubrum \&c. 

Lusor victor qui primus duxit tres pedites ad nodum aequi coloris.

}

\renewcommand{\rulest}{Regulae } 
\renewcommand{\rules}{ 
Duc peditum tuorum unum in quamvis directionem sequens aut viam circuli aut vias stellae, quoad per regulas potes occupare aediculum. 

\myskip

In quodlibet aediculo libero, ubi  duae lineae conveniunt, tibi licet de via recta declinare sine retentione. 

\myskip

At numquam tibi licet transcendere nec impedimenta nec pedites alios.

\myskip

Potes vero ingredi in aediculum occupatum:

\myskip

Se ibi stat impedimentum nigrum, id pone in aediculo libero ad libitum. 

Se ibi iam stat pedes alius, pedites positiones suas mutare debent.

Se ibi sunt plures pedites (non potest fieri, nisi initio ludi), elige unum mutandum.

\myskip

Ita potes etiam mutare positiones duorum peditum tuorum.

\myskip

Bis eundem ne cana; eundem motum non licet iterum.

\myskip

Pedes progressus ad finem ludo abit. Pone eum in centrum.

Accipe per illo impedimentum canum et colloca in aediculo, quoad tibi paret. 

\myskip

Quando captas tale impedimentum canum, remove a tabula.

\myskip

Lusor  qui primus tres pedites ad finem duxit ludo victor.

\myskip

Gesta, non verba!
}
